%Template file for Scientific Computation project 1 discussion and figures
\documentclass{article}
\usepackage[a4paper, margin=1in]{geometry}
\title{Scientific Computation Project 1}
\usepackage{minted}
\usepackage{amsmath}
\usepackage{amssymb}
\newcommand{\trm}{\textrm}
\newcommand{\pa}{\partial}
\author{\emph{Your college ID here}}

\usepackage{graphicx}

\begin{document}

\maketitle

%---------------- Part 1  -------------------
\hrule
\hrule

\subsection*{Part 1}


\subsubsection*{1.}
%Place your discussion for question 1 here
1(a) Method 1 is a simple linear search,
where it iterates through the list L and checks if x is present.
Method 2 sorts the list of IDs using merge sort first if the list is not sorted, 
then performs a binary search to find the element. \\
1(b) Method 1: O(mn) since it's a simple linear search with m multiple times using
the results from lecture notes. \\
Method2: For the worst case, we need to sort the list with complexity O(nlog2n),
then apply binary search with complexity O(mlog2n).
Hence, the total complexity is O((m+n)log2n). 
2. To compare the performance of method1 and method2, we can fix either the size of
m or n and compare O(mn) with O((m+n)log2n). 
Firstly, we fix the value of n (I use n=100 and 10000 for the code), and results are shown
in Figure 1 and Figure 2. The results show that for small m (m < 150 in the graphs),
method2 is a little bit slower than method1. This can be seen by the mathematical
comparsion of O(mn) and O((m+n)log2n), or by the consideration of the cost of the merge sort
in method2. As the size of m increases, method1 takes much longer time than method2 shown
in the graphs. This is because O(mn) is much bigger than O((m+n)log2n) if we fix the value of
n for large m. Hence, the results shown in the graphs agree with our theoratical results.
Now, we fix the value of m (I use m=10 and 1000 for the code), and the result is shown
in Figure 3 and Figure 4. For m=10 (i.e. m is small), the results show that there is not much
difference between method1 and method2 as expected for small n. As the size of n increases,
method2 becomes much slower than method1 shown in the graph, since O((m+n)log2n) is much bigger than O(mn)
in this case. For m=1000 (i.e. m is large), as the increase of n, method2 is much faster than method1
as expected from the comparison of O(mn) and O((m+n)log2n). Hence, the results shown in the graphs also agree
with our theoratical reslts.  






\subsubsection*{2.}
%Place your discussion for question 2 here


\begin{figure}[h!]
\centering
%Uncomment line below to display figure saved as fig1.png
\includegraphics[width=0.8\textwidth]{fig1.png}

\caption{Figure 1: Add figure description here}
\label{fig1}
\end{figure}

%Add additional figure if needed



%---------------- End Part 1 -------------------

\vspace{0.25in}

%---------------- Part 2  -------------------
\subsection*{Part 2}

\subsubsection*{2.}

%Place your discussion for question 2 here}
%---------------- End Part 2 -------------------


\hrule
\hrule



%---------------- End document -------------------


\end{document}
